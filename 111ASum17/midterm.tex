\documentclass[11pt]{article}
\usepackage{amsmath}
\usepackage{centernot}
\usepackage{listings}
\usepackage{amssymb}
\usepackage{array}
\usepackage{geometry}
\usepackage{amsthm}
\usepackage{mathtools}
\usepackage[mathscr]{euscript}
\usepackage{relsize}
\usepackage{mathrsfs}
\usepackage{tikz}
\usetikzlibrary{matrix,arrows}
\usepackage{blindtext}
\usepackage[utf8]{inputenc}

\usepackage{url}


\newtheorem{defn}{Definition}
\newtheorem{thm}{Theorem}
\newtheorem{cor}{Corollary}
\newtheorem{ex}{Example}
\newtheorem{lm}{Lemma}
\newtheorem{pr}{Proposition}
\newtheorem{nt}{Note}
\newtheorem{conj}{Conjecture}
\newtheorem{rem}{Remark}

\DeclareMathOperator{\Spec}{Spec}
\DeclareMathOperator{\Proj}{Proj}
\DeclareMathOperator{\Div}{Div}
\DeclareMathOperator{\Symb}{Symb}
\DeclareMathOperator{\Sym}{Sym}
\DeclareMathOperator{\SL}{SL}
\DeclareMathOperator{\Meas}{Meas}
\DeclareMathOperator{\Dist}{Dist}
\DeclareMathOperator{\Cont}{Cont}
\DeclareMathOperator{\Step}{Step}
\DeclareMathOperator{\Ind}{Ind}
\DeclareMathOperator{\End}{End}
\DeclareMathOperator{\Sp}{Sp}
\DeclareMathOperator{\Mel}{Mel}
\DeclareMathOperator{\spec}{sp}
\DeclareMathOperator{\Res}{Res}


\newcommand{\ds}{\displaystyle}
\newcommand{\N}{\mathbb{N}}
\newcommand{\suml}{\ds\sum_{n=1}^\infty}
\newcommand{\sums}{\ds\sum_{k=n}^m}
\newcommand{\ms}{\medskip}
\newcommand{\R}{\mathbb{R}}
\newcommand{\C}{\mathbb{C}}
\newcommand{\Prj}{\mathbb{P}}
\newcommand{\F}{\mathbb{F}}
\newcommand{\D}{\mathbb{D}}
\newcommand{\W}{\mathbb{W}}
\newcommand{\Z}{\mathbb{Z}}
\newcommand{\Q}{\mathbb{Q}}
\newcommand{\T}{\mathbb{T}}
\newcommand{\We}{\mathcal{W}}
\newcommand{\m}{\mathfrak{m}}
\newcommand{\p}{\mathfrak{p}}
\newcommand{\pa}{\mathfrak{P}}
\newcommand{\q}{\mathfrak{q}}
\newcommand{\ga}{\mathfrak{a}}
\newcommand{\qa}{\mathfrak{Q}}
\newcommand{\OX}{\mathscr{O}}
\newcommand{\OK}{\mathscr{O}_K}
\newcommand{\OL}{\mathscr{O}_L}
\newcommand{\OF}{\mathscr{O}_F}
\newcommand{\Frob}{\text{Frob}}
\newcommand{\Gal}{\text{Gal}}
\newcommand{\Cy}{\mathcal{C}}
\newcommand{\GL}{\text{GL}}
\newcommand{\tr}{\text{tr}}
\newcommand{\Cl}{\text{Cl}}
\newcommand{\A}{\mathbb{A}}
\newcommand{\U}{\mathcal{U}}
\newcommand{\OFv}{\mathscr{O}_{F,v}}
\newcommand{\OFp}{\mathscr{O}_{F,\p}}
\newcommand{\val}{\text{val}}
\newcommand{\Hom}{\text{Hom}}
\newcommand{\Sf}{\mathscr{F}}
\newcommand{\Sg}{\mathscr{G}}
\newcommand{\im}{\text{im}}
\newcommand{\diagram}{\begin{tikzpicture}[description/.style={fill=white,inner sep=2pt}]
\matrix (m) [matrix of math nodes, row sep=3em,
column sep=2.5em, text height=1.5ex, text depth=0.25ex]}
\newcommand{\id}{\text{id}}
\newcommand{\Spe}{\text{Spe}}

\setlength{\parskip}{.25em}



\usepackage{fancyhdr}
\setlength{\headheight}{15.2pt}
\pagestyle{fancy}
\fancyhf{}
\renewcommand{\headrulewidth}{0pt}
\lfoot{}
\rhead{Name: $\rule{4cm}{0.15mm}$}
\cfoot{\thepage}


\begin{document}
\begin{center} \Large\textbf{Midterm} \end{center}

\noindent This exam has 11 pages and 8 problems. Make sure that your exam has all 11 pages and that your name is on every page. 

\noindent Put your student ID under your name on the first page. 

\noindent You must show your work and justify your answers to receive full credit unless otherwise stated. 

\noindent If you need more space, use the pack of the pages; clearly indicate when you have done this. 

\noindent You may not use books, notes, or any calculator on this exam; nor any technology that can serve as such.

\begin{enumerate}

\item (20 points) One word or number answers. You will receive 2 points for each correct answer and 0 points for an incorrect answer. No justification is necessary.
\begin{enumerate} \item[(1)]  Is the rational numbers with the binary operation of multiplication a group?

Answer: $\rule{2cm}{0.15mm}$

\item[(2)] Is the positive real numbers with the binary operation of multiplication a group?

Answer: $\rule{2cm}{0.15mm}$

\item[(3)]  Is the integers with the binary operation of multiplication a group?

Answer: $\rule{2cm}{0.15mm}$

\item[(4)] What is the order of the element $(123)(45)$ of $\text{Sym}(5)$?

Answer: $\rule{2cm}{0.15mm}$

\item[(5)] What is the order of the element $(12345)(678)$ of $\text{Sym}(8)$?

Answer: $\rule{2cm}{0.15mm}$

\item[(6)] In $\text{Sym}(5)$, what is a cycle decomposition of $(123)(13)$?

Answer: $\rule{2cm}{0.15mm}$

\item[(7)] What is the order of $\overline{98}$ in $\Z/140\Z$?

Answer: $\rule{2cm}{0.15mm}$

\item[(8)] Does a subgroup of $\text{Sym(5)}$ have order 7?

Answer: $\rule{2cm}{0.15mm}$

\item[(9)] Does a subgroup of $\text{Sym(5)}$ have order 6?

Answer: $\rule{2cm}{0.15mm}$

\item[(10)] If $H$ is a subgroup of a group $G$ and $aH = bH$, is $a^{-1}b$ an element of $H$?

Answer: $\rule{2cm}{0.15mm}$

\end{enumerate}
\pagebreak

\item Consider the group $G = \Z/36\Z$. For this problem no justification is necessary.
\begin{enumerate} \item[(a)] (5 points) What are all the generators of $G$?
\vspace{6cm}
\item[(b)] (5 points) List all the subgroups of $G$.
\vspace{10cm}
\end{enumerate}
\pagebreak

\item Consider the group $G = \text{Sym}(7)$, and let $\sigma\in G$ be the permutation:
$$1\mapsto 3; 2\mapsto 5; 3\mapsto 7; 4\mapsto 6; 5\mapsto 2; 6\mapsto 1; 7\mapsto 4$$
\begin{enumerate}\item[(a)] (5 points) Write down a cycle decomposition of $\sigma$.
\vspace{7cm}
\item[(b)] (5 points) Write $\sigma$ as a product of transpositions. What is the sign of $\sigma$?
\vspace{7cm}
\end{enumerate}
\pagebreak

\item \begin{enumerate}\item[(a)] (5 points) What is the definition of a group?
\vspace{5cm}

\item[(b)] (5 points) Let $(G,*)$ and $(H, \square)$ be two groups. Consider the cartesian product of $G$ and $H$,
$$G\times H = \{(g,h): g\in G, h\in H\}$$
with the binary operation
$$\triangle: (G\times H)\times (G\times H) \longrightarrow G \times H$$
given by
$$(g_1,h_1)\triangle (g_2,h_2) = (g_1*h_1, g_2\square h_2)$$
Prove that $(G\times H, \triangle)$ is a group.
\vspace{8cm}
\end{enumerate}
\pagebreak

\item \begin{enumerate} \item[(a)] (5 points) Let $(G,*)$ and $(H,\square)$ be two groups. What is the definition of a group homomorphism from $G$ to $H$?

\vspace{4cm}

\item[(b)] (5 points) Let $\GL_2(\R) = \left\{\begin{pmatrix} a & b \\ c & d\end{pmatrix}: a,b,c, d\in \R, ad-bc\not=0\right\}$ be the set of invertible 2 by 2 matrices, and let $\R^\times = \R - \{0\}$ be the set of nonzero real numbers. Prove that the determinant function
$$\det:\GL_2(\R)\longrightarrow \R^\times$$
$$\det\left(\begin{pmatrix} a & b\\c & d\end{pmatrix}\right) = ad-bc$$
is a group homomorphism from $\GL_2(\R)$, where the group operation is matrix multiplication, to $\R^\times$, where the group operation is multiplication of numbers.
\pagebreak

\item[(c)] (5 points) Let $f:G\rightarrow H$ be a group homomorphism from a group $G$ to a group $H$. What is the definition of the kernel of $f$ and the image of $f$?
\vspace{6cm}

\item[(d)] (5 points) What is the image and kernel of the determinant function from part (b)? To get full credit for this problem you must state what the image and kernel are as well as prove it.
\pagebreak
\end{enumerate}

\item Let $G, H$ be two groups, and let $f:G\rightarrow H$ be a group homomorphism from $G$ to $H$. \begin{enumerate} \item[(a)] (5 points) What is the definition of a subgroup of $G$?
\vspace{6cm}

\item[(b)] (5 points) Prove that the kernel of $f$ is a subgroup of $G$.
\pagebreak
\item[(c)] (5 points) What is the definition of a normal subgroup of $G$?
\vspace{5cm}
\item[(d)] (5 points) Prove that the kernel of $f$ is a normal subgroup of $G$.
\pagebreak
\end{enumerate}

\item (10 points) Let $m,n\in\N$. Consider the function
$$\begin{array}{rcl} f:\Z/mn\Z &\longrightarrow &\Z/n\Z\\
						\overline{a} & \longmapsto &\overline{a}\end{array}$$
Prove that $f$ is well-defined. That is, prove that $f(\overline{a})$ does not depend on the representative of the equivalence class $\overline{a}$ chosen. Prove that $f$ is a group homomorphism. What is the kernel of $f$ and what is the image of $f$?
\pagebreak

\item Let $G$ be a finite group, and let $H, K$ be subgroups of $G$ such that $K\subset H\subset G$. 

\begin{enumerate} \item[(a)] (10 points) Recall that the index of a subgroup, $N$, in $G$ is the number of left cosets of $N$ in $G$, and is denoted by $[G:N]$. Show that $[G:K] = [G:H][H:K]$.

\pagebreak

\item[(b)] (10 points) Let $N$ be a subgroup of a finite group $G$. A set of coset representatives of $N$ in $G$, is a collection of elements of $G$, $\{g_1,g_2,\ldots, g_k\}\subset G$, such that
$$G = \bigcup_{i = 1}^k g_i N \text{ and } g_iN\cap g_jN=\emptyset\text{ if }i\not=j$$
Let $\{a_1,a_2,\ldots, a_n\}\subset G$ be a set of coset representatives for $H$ in $G$, and let $\{b_1,b_2,\ldots, b_m\}\subset H$ be a set of coset representatives for $K$ in $H$. Prove that the set 
$$\{a_ib_j : 1\leq i\leq n, 1\leq j\leq m\}$$
is a set of coset representatives for $K$ in $G$.
\pagebreak
\end{enumerate}


\end{enumerate}




\end{document}